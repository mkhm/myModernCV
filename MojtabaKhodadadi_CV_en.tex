%% start of file `template.tex'.
%% Copyright 2006-2013 Xavier Danaux (xdanaux@gmail.com).
%
% This work may be distributed and/or modified under the
% conditions of the LaTeX Project Public License version 1.3c,
% available at http://www.latex-project.org/lppl/.

\documentclass[10pt,a4paper,sans]{moderncv}        % possible options include font size ('10pt', '11pt' and '12pt'), paper size ('a4paper', 'letterpaper', 'a5paper', 'legalpaper', 'executivepaper' and 'landscape') and font family ('sans' and 'roman')
%\AtBeginDocument{\hypersetup{pdfborder = 1 1 1,linkcolor=red}}

% moderncv themes
\moderncvstyle{classic}                             % style options are 'casual' (default), 'classic', 'oldstyle' and 'banking'
\moderncvcolor{blue}                               % color options 'blue' (default), 'orange', 'green', 'red', 'purple', 'grey' and 'black'
%\renewcommand{\familydefault}{\sfdefault}         % to set the default font; use '\sfdefault' for the default sans serif font, '\rmdefault' for the default roman one, or any tex font name
%\nopagenumbers{}                                  % uncomment to suppress automatic page numbering for CVs longer than one page
\usepackage[bottom,norule]{footmisc}
\usepackage[utf8]{inputenc}
\usepackage{fancyhdr}

% character encoding
%\usepackage[utf8]{inputenc}                       % if you are not using xelatex ou lualatex, replace by the encoding you are using
%\usepackage{CJKutf8}                              % if you need to use CJK to typeset your resume in Chinese, Japanese or Korean

% adjust the page margins
\usepackage[left = 1.2cm, right = 1.2cm, top = 1.5cm, bottom = 1.5cm]{geometry}
%\usepackage[scale=0.8]{geometry}
\setlength{\hintscolumnwidth}{2.5cm}                % if you want to change the width of the column with the dates
\setlength{\makecvtitlenamewidth}{10cm}           % for the 'classic' style, if you want to force the width allocated to your name and avoid line breaks. be careful though, the length is normally calculated to avoid any overlap with your personal info; use this at your own typographical risks...

% personal data
\name{Mojtaba}{Khodadadi}
%\title{Curriculum Vitae}                               % optional, remove / comment the line if not wanted
\address{Room A, ICTP, Strada Costiera, 11}{I - 34151, Trieste, Italy}{}% optional, remove / comment the line if not wanted; the "postcode city" and "country" arguments can be omitted or provided empty
\phone[mobile]{+98~(913)~716~1810}                   % optional, remove / comment the line if not wanted; the optional "type" of the phone can be "mobile" (default), "fixed" or "fax"
%\phone[fixed]{+2~(345)~678~901}
%\phone[fax]{+3~(456)~789~012}
\email{mkh.240@gmail.com}                               % optional, remove / comment the line if not wanted
\homepage{users.ictp.it/\~mkhodada}                         % optional, remove / comment the line if not wanted
\social[linkedin]{mkh240}                        % optional, remove / comment the line if not wanted
%\social[twitter]{jdoe}                             % optional, remove / comment the line if not wanted
\social[github]{mkhm}                              % optional, remove / comment the line if not wanted
\extrainfo{ \httplink[stackoverflow]{www.stackoverflow.com/users/3454902/mkh-240}}  %  http://stackoverflow.com/cv/mkh240                      %\httplink{www.github.com/me}\\}
%\extrainfo{additional information}                 % optional, remove / comment the line if not wanted
%\photo[64pt][0.4pt]{picture}                       % optional, remove / comment the line if not wanted; '64pt' is the height the picture must be resized to, 0.4pt is the thickness of the frame around it (put it to 0pt for no frame) and 'picture' is the name of the picture file
%\quote{Complex Systems are Everywhere and I am Interested in Analyzing them.}                                 % optional, remove / comment the line if not wanted

% to show numerical labels in the bibliography (default is to show no labels); only useful if you make citations in your resume
%\makeatletter
%\renewcommand*{\bibliographyitemlabel}{\@biblabel{\arabic{enumiv}}}
%\makeatother
%\renewcommand*{\bibliographyitemlabel}{[\arabic{enumiv}]}% CONSIDER REPLACING THE ABOVE BY THIS

% bibliography with mutiple entries
%\usepackage{multibib}
%\newcites{book,misc}{{Books},{Others}}
%----------------------------------------------------------------------------------
%            content
%----------------------------------------------------------------------------------
%\usepackage{xcolor}
%\usepackage{xcolor,soul,lipsum} 
%\usepackage[hidelinks]{hyperref}  
%\usepackage{hyperref} 
%\hypersetup{colorlinks=false,linkbordercolor=red,linkcolor=green,pdfborderstyle={/S/U/W 1}} 
%\usepackage[hidelinks=true]{hyperref}
%\usepackage[colorlinks=true, urlcolor=blue, pdfborder={0 0 0}]{hyperref}
\newcommand{\comment}[1]{}
\definecolor{darkblue}{rgb}{0.0,0.0,0.7}
\definecolor{refsection}{rgb}{0.0,0.5,0.0}
\definecolor{projectaward}{rgb}{0.0,0.0,0.7}
\definecolor{darkred}{rgb}{0.7,0.0,0.0}
\definecolor{darkred2}{rgb}{0.85,0.0,0.0}
\newcommand{\MYhref}[3][blue]{\href{#2}{\color{#1}{#3}}}
%example: \MYhref[darkred]{http://www.uni-oldenburg.de/}{Oldenburg University}
\newcommand\Colorhref[3][cyan]{\href{#2}{\small\color{#1}#3}}
\newcommand{\myhy}[2]{\href[#1]{\color{green}\setulcolor{red}\ul{#2}}} 
\newcommand{\moreinfo}{\nolinkurl{more} \nolinkurl{info}}
\newcommand{\invitationLetter}{\nolinkurl{invitation} \nolinkurl{letter}}
 \newcommand{\gradingSystem}{\nolinkurl{grading} \nolinkurl{system}}
 

\newcommand{\superscript}[2][darkred2]{{\color{#1}$^{#2}$}}
\newcommand{\urlfootnote}[2]{\footnotetext[#1]{\href{#2}{#2}}}
%\urlfootnote{1}{http://www.shazam.com}
\newcommand{\myurlfootnote}[4][blue]{\footnotetext[#2]{\MYhref[#1]{#3}{#4}}}
% equals to bellow:
%\newcommand{\myurlfootnote}[4][blue]{\footnotetext[#2]{\href{#3}{\color{#1}{#4}}}}
%example: \myurlfootnote[darkred]{1}{http://www.uni-oldenburg.de/}{Oldenburg University}

%\renewcommand{\headrulewidth}{0.4pt}% Default \headrulewidth is 0.4pt
%\renewcommand{\footrulewidth}{0.4pt}% Default \footrulewidth is 0pt
%\def\mydoublerule#1#2#3{%%
%  \hrule width#1 height#2 \vskip#2
%  \hrule width#1 height#3 
%}
\usepackage{xcolor}
\def\SPSB#1#2{\rlap{\textsuperscript{\textcolor{red}{#1}}}\SB{#2}}
\def\SP#1{\textsuperscript{\textcolor{red}{#1}}}
\def\SB#1{\textsubscript{\textcolor{blue}{#1}}}
\comment{
sometext\SPSB{1}{2} more text
sometext\SP{1} more\SB{2} text\SP{1}
}



\begin{document}
%\hypersetup{ colorlinks = true, %Colours links instead of ugly boxes 
%urlcolor = blue, %Colour for external hyperlinks 
%linkcolor = blue, %Colour of internal links 
%citecolor = red %Colour of citations 
%}
%\hypersetup{ colorlinks, linkcolor={rad}, citecolor={blue}, urlcolor={blue} }

%\begin{CJK*}{UTF8}{gbsn}                          % to typeset your resume in Chinese using CJK
%-----       resume       ---------------------------------------------------------
\makecvtitle

\vspace*{-1.2cm} %reduce that giant whitespace

\section{Research Interests}
%\cvlistdoubleitem{Complex Systems}{Online Social Media \& Networks Analysis}
%\cvlistdoubleitem{Computational Social Science}{Application of Machine Learning in Data Analysis}
%\cvlistdoubleitem{Stochastic Processes}{Econophysics}
\cvitem{}{Application of Machine Learning in Data Analysis, Computational Social Science, Online Social Media \& Networks Analysis, Social Physics, The Physics of Data, Complex Systems, Stochastic Processes, Statistical Physics of \{Complex Networks, Social Systems, Financial Markets\}, Econophysics, Computational Finance, Collective Phenomena in Complex Systems, Computational Neuroscience}
%\cvitem{General}{Data Analysis, Complex Systems, Stochastic Processes, Complex Networks}
%\cvitem{}{Complex Systems, Stochastic Processes, Complex Networks}

\section{Education}
\cventry{2015 -- 2016}{ICTP Postgraduate Diploma Programme in Physics}{\MYhref[darkred]{http://ictp.it/}{International Centre for Theoretical Physics (ICTP)}}{Trieste, Italy}{}{}
\comment{International School for Advanced Studies (SISSA) \& International Centre for Theoretical Physics (ICTP)}
%\cventry{2013 -- 2015}{M.Sc. Condensed Matter Physics}{Isfahan University of Technology (IUT)}{Isfahan, Iran}{\newline{} GPA: \textit{19.06/20 (4.00/4.00)}}{Supervisor: \MYhref[darkblue]{http://shahbazi.iut.ac.ir/}{Dr. Farhad Shahbazi} from \MYhref[darkred]{http://iut.ac.ir/en/}{Isfahan University of Technology} \& \MYhref[darkblue]{http://www.uni-oldenburg.de/en/iro/current-visiting-researchers/alexander-von-humboldt-fellows/faculty-v/institute-of-physics/prof-dr-m-reza-rahimi-tabar/}{Prof. Mohammad Reza Rahimi Tabar} from \MYhref[darkred]{http://www.sharif.edu/web/en/}{Sharif University of Technology} and \MYhref[darkred]{http://www.uni-oldenburg.de/}{Oldenburg University}. \newline{} Master's thesis title: "Investigation of Epileptic Seizures by Analyzing EEG Data".
\cventry{2013 -- 2015}{M.Sc. Condensed Matter Physics with Highest Distinction}{Isfahan University of Technology (IUT)}{Isfahan, Iran}{}{
\begin{itemize}
\item GPA: \textit{19.15/20 (\%96, 4.00/4.00)}, thesis grade: \textit{19.5/20 (\%98, 4.00/4.00)}.
\item Master's thesis: "Investigation of Epileptic Seizures by Analyzing EEG Data using Machine Learning".  \MYhref[blue]{https://users.ictp.it/~mkhodada/MasterThesisAbstract.pdf}{\nolinkurl{abstract}}
\item Supervisors: \MYhref[darkblue]{http://shahbazi.iut.ac.ir/}{Dr. Farhad Shahbazi} from \MYhref[darkred]{http://iut.ac.ir/en/}{Isfahan University of Technology} \& \MYhref[darkblue]{http://www.uni-oldenburg.de/en/iro/current-visiting-researchers/alexander-von-humboldt-fellows/faculty-v/institute-of-physics/prof-dr-m-reza-rahimi-tabar/}{Prof. Mohammad Reza Rahimi Tabar} from \MYhref[darkred]{http://www.sharif.edu/web/en/}{Sharif University of Technology} and \MYhref[darkred]{http://www.uni-oldenburg.de/}{Oldenburg University}. %Carl von Ossietzky University
\end{itemize}
}

% arguments 3 to 6 can be left empty
%\cventry{2009 -- 2013}{B.Sc. Physics}{Isfahan University of Technology (IUT)}{Isfahan, Iran}{\newline{} GPA: \textit{17.54/20 (3.78/4.00)}, and Major GPA: \textit{17.82/20 (3.85/4.00)}}{} 
\cventry{2009 -- 2013}{B.Sc. Physics with Distinction}{Isfahan University of Technology (IUT)}{Isfahan, Iran}{}{
\begin{itemize}
\item GPA: \textit{17.54/20 (\%88, 3.78/4.00)}, and Major GPA: \textit{17.82/20 (\%90, 3.85/4.00)}.
\end{itemize}
} 
%\cventry{2005 -- 2009}{Diploma Mathematics \& Physics}{Saadi High School \& Adle Pre-University School}{Isfahan, Iran}{\newline{} Diploma GPA: \textit{19.69/20 (4.00/4.00)}, and Pre-University GPA: \textit{19.51/20 (4.00/4.00)}}{}
\cventry{2005 -- 2009}{Diploma Mathematics \& Physics}{Saadi High School \& Adle Pre-University School}{Isfahan, Iran}{}{
\begin{itemize}
\item Diploma GPA: \textit{19.69/20 (\%98, 4.00/4.00)}, and Pre-University GPA: \textit{19.51/20 (\%98, 4.00/4.00)}.
%\item Graduated first in a class of 700 students at age 17.
\end{itemize}
}


%\section{Master thesis}
%\cvitem{title}{\emph{Title}}
%\cvitem{supervisors}{Supervisors}
%\cvitem{description}{Short thesis abstract}

\comment{
\cventry{year--year}{Job title}{Employer}{City}{}{General description no longer than 1--2 lines.\newline{}%
Detailed achievements:%
\begin{itemize}%
\item Achievement 1;
\item Achievement 2, with sub-achievements:
  \begin{itemize}%
  \item Sub-achievement (a);
  \item Sub-achievement (b), with sub-sub-achievements (don't do this!);
    \begin{itemize}
    \item Sub-sub-achievement i;
    \item Sub-sub-achievement ii;
    \item Sub-sub-achievement iii;
    \end{itemize}
  \item Sub-achievement (c);
  \end{itemize}
\item Achievement 3.
\end{itemize}}
}



\section{\textbf{Honors and Awards}}
\subsection{Computer-related:}
\cventry{2013}{Merit-Based Admission Offer}{}{to full-time Master Program in both \textbf{Software Engineering and AI} without entrance exam, School of Computer Engineering, Iran University of Science and Technology}{Tehran, Iran}{}
\cventry{2010}{Honored as Outstanding Researcher Activity}{SBM Project}{\comment{Dr. Hamedani Golshan, former }President of IUT}{Isfahan, Iran}{} % \superscript{1}
%\cventry{2010}{\textmd{Isfahan Province Education Organization Award}}{SBM Project}{Presidents' Organization}{Isfahan, Iran}{}
%\cventry{2010}{\textmd{Research Council of Isfahan Province Education Organization Award}}{SBM Project}{Presidents' Organization}{}{} %Isfahan, Iran
\cventry{2010}{Institute of Steel Employee Support \& Retirement Fund Award}{SBM Project}{Chairman \& CEO}{Tehran, Iran}{}
\cventry{2009}{Ranked 2nd in 11th Iran's National Youth Kharazmi Award with the Statuette}{for ''Smart Bandwidth Manager (SBM)'' Project}{Minister of Science, Research and Technology \& Minister of Education}{Iran}{}
\subsection{Physics-related:}
\cventry{2010 \& 2015}{Isfahan University of Technology Dean Award with the Statuette}{}{\comment{Dr. Modarres-Hashemi, }President of IUT}{Isfahan, Iran}{} % Mahmood Modarres-Hashemi
\cventry{2014 \& 2015}{Ranked 1st, out of 58 Students}{in all Programs of M.Sc. Physics according to GPA}{IUT}{Isfahan, Iran}{}
\cventry{2014 \& 2015}{Ranked 1st, out of 34 M.Sc. Students}{in Condensed Matter Physics Program according to GPA}{IUT}{}{} %Isfahan, Iran
\cventry{2013}{Ranked 15th in 18th Iran's National Collegiate Scientific Olympiad}{}{among all B.Sc Physics Students of Nation}{Iran}{}
\cventry{2013}{Merit-Based Admission Offer}{}{to full-time Master Program in \textbf{Physics} without Entrance exam, IUT}{Iran}{} % Isfahan, Iran
\cventry{2013}{Ranked 4th, out of 61 in Undergraduate Physics program}{}{according to GPA, IUT}{Isfahan, Iran}{}
%\cventry{2010 -- 2015}{Selected as an Exceptional Talent in Physics Department}{}{IUT}{Isfahan, Iran}{}


%\footnotetext[1]{Isfahan University of Technology}


\section{\textbf{Fellowships \& Grants}}
\cventry{2015 -- 2016}{ICTP Scholarship and Travel Grant}{}{International Centre for Theoretical Physics}{Trieste, Italy}{}
\cventry{2013 \& 2015}{Iran's National Elites Foundation 2-year Graduate Fellowship}{}{}{Government of Iran}{}
\cventry{2010 \& 2015}{University Outstanding Student Scholarship}{}{\comment{Dr. Hamedani Golshan, former President of }Isfahan University of Technology}{Isfahan, Iran}{}
\cventry{2014}{Isfahan University of Technology Research Grant}{for Master Thesis}{Isfahan}{Iran}{}
\cventry{2009 -- 2013}{Iran's National Elites Foundation 4-year Undergraduate Fellowship}{}{}{Government of Iran}{}
\cventry{2011}{Isfahan Science \& Technology Town (ISTT) Research Grant}{for ''Electronic Content Management System (eCMS)'' Project}{Association of Creativity Efflorescence}{Isfahan, Iran}{}
\cventry{2010}{Iran's National Elites Foundation Research Grant}{for ''Smart Bandwidth Manager (SBM)'' Project}{Vice President for Science and Technology}{Government of Iran}{}

%\vspace{1ex}
%\mydoublerule{\linewidth}{1pt}{3pt}
%\vspace{1ex}

%\vspace{2ex}
%\hrule 


\section{Relevant Courses}
\cvitem{Postgraduate Program, ICTP}{Statistical Mechanics (E, \%100), Numerical Methods (E, \%100), Numerical Methods II (E, \%100). \newline{} E: Excellent. \MYhref[blue]{http://diploma.ictp.it/courses/grading-system.aspx}{\gradingSystem}}
\cvitem{M.Sc., IUT}{Computational Physics (20, \%100), Advanced Statistical Mechanics (19.1, \%96), \newline{} Advanced Statistical Mechanics II (20, \%100). \MYhref[blue]{https://en.wikipedia.org/wiki/Academic_grading_in_Iran}{\gradingSystem}} % \newline{} Grades are out of 20.
\cvitem{B.Sc., IUT}{Computational Physics (20 out of 20, \%100), Statistical Mechanics (20, \%100), Biophysics (20, \%100). \MYhref[blue]{https://en.wikipedia.org/wiki/Academic_grading_in_Iran}{\gradingSystem}} % \newline{} Grades are out of 20.
%\cvitem{Postgraduate Program, ICTP}{Statistical Mechanics (grade: E [Excellent]), Numerical Methods (E), Numerical Methods II (E). \newline{} \MYhref[blue]{http://diploma.ictp.it/courses/grading-system.aspx}{\gradingSystem}}
%\cvitem{M.Sc., IUT}{Computational Physics (20), Advanced Statistical Mechanics (19.1), Advanced Statistical Mechanics II (20).} % \newline{} Grades are out of 20.
%\cvitem{B.Sc., IUT}{Computational Physics (20 out of 20), Statistical Mechanics (20), Advanced Statistical Mechanics II (20).} % \newline{} Grades are out of 20.



\section{Research Experience}
%\subsection{Physics related}
\comment{
\cventry{2013 -- 2014}{Graduate Research Assistant}{under supervision of \MYhref[darkblue]{http://salari.iut.ac.ir/}{Dr. V. Salari}}{Physics Dept., Isfahan University of Technology}{Isfahan, Iran}{Fields of research:
\begin{itemize}
\item Working on ion channel data to make a classification between different potential differences by use of correlation matrix method.
\item Estimating number of molecular vibrational modes contributing thermal energy to pigment activation using nonlinear least squares method.
\end{itemize}
}
}
\cventry{2014 -- 2015}{Graduate Researcher (Master's Thesis)}{under supervision of \MYhref[darkblue]{http://shahbazi.iut.ac.ir/}{Dr. F. Shahbazi} and \MYhref[darkblue]{http://www.uni-oldenburg.de/en/iro/current-visiting-researchers/alexander-von-humboldt-fellows/faculty-v/institute-of-physics/prof-dr-m-reza-rahimi-tabar/}{Prof. M.R.R. Tabar}}{Physics department, Isfahan University of Technology and Sharif University of Technology}{Iran. \hfill \MYhref[blue]{https://users.ictp.it/~mkhodada/MasterThesisAbstract.pdf}{\nolinkurl{abstract}}}{Fields of research: EEG Data Analysis, Machine Learning, Stochastic Processes, Complex Systems.
\begin{itemize}
\item Identifying epilepsy seizures focus by correlation matrix and stochastic processes methods.
\item Investigating the possibility of seizure prediction using hierarchical clustering of correlation matrices of different parts of brain over time.
\end{itemize}
}

%\subsection{Computer related}
\cventry{2011 -- 2013}{\comment{Undergraduate }Researcher}{Isfahan Science \& Technology Town (ISTT)}{Isfahan}{Iran}{
I was member of \textit{Association of Creativity Efflorescence} at ISTT.
\begin{itemize}
\item Implementing an Integrated \textit{''Electronic Content Management System (eCMS)''}.
\item Upgrade eLibrary to eCMS.
\end{itemize}
}
\cventry{2007 -- 2011}{Researcher}{IT Center, Isfahan University of Technology}{Isfahan}{Iran}{
\begin{itemize}
\item Designing and Implementing \textit{''Smart Bandwidth Manager (SBM)''} core.
\item Developing, Testing and Debugging SBM v2.2.1 .
\end{itemize}
}
\cventry{2010}{Researcher}{Research Projects Office, Isfahan Science \& Technology Town (ISTT)}{Isfahan}{Iran}{
\begin{itemize}
\item Developing, Testing and Debugging SBM v2.1.1 .
\end{itemize}
}


\section{Publications}
\cventry{}{\textnormal{Mojtaba Khodadadi, Farhad Shahbazi, }"Classification of Epilepsy Focus in Interictal and Ictal States"}{accepted for 20-minute presentation at \textbf{CCS'15 (Conference On Complex Systems)}}{Sep 26 - Oct 2, 2015 at Tempe, Arizona, USA}{\hfill \MYhref[blue]{https://users.ictp.it/~mkhodada/CCS2015-Invitation-Letter.pdf}{\invitationLetter}}{}
\cventry{}{\textnormal{Mojtaba Khodadadi, et. al., }"PyMarkov: A Python Module for Markovian Stochastic Processes"}{In preparation}{}{}{}
\cventry{}{\textnormal{Mojtaba Khodadadi, et. al., }"PyCorrState: A Python Package for Correlation States Extraction"}{In preparation}{}{}{}
\cventry{}{\textnormal{Mojtaba Khodadadi, et. al., }''Identifying Epilepsy Focus by Correlation Matrix Method"}{In preparation}{}{}{}
\cventry{}{\textnormal{Mojtaba Khodadadi, et. al., }"Identifying States of an Epilepsy Seizure"}{In preparation}{}{}{}


\section{\textbf{Invited Lectures}}
\cventry{Winter 2015}{Scientific Computing with Python: An Introduction to Data Analysis \& High Performance Computing (16 hours) \textnormal{, \textbf{\color{red} at age 23}, for graduate students of Physics}}{Department of Physics, Isfahan University of Technology}{Isfahan}{Iran. \MYhref[blue]{http://www.slideshare.net/MojtabaKhodadadi/python-courseen-44968289}{\moreinfo} \hfill \MYhref[blue]{https://github.com/mkhm/python-scientific-computing-course-2015}{\nolinkurl{github}}}{} %More info
%Further information about topics: \MYhref[cyan]{http://www.slideshare.net/MojtabaKhodadadi/python-courseen-44968289}{\nolinkurl{link}}.
%I was invited by \textit{"Department of Physics"} of Isfahan University of Technology to teach Scientific Computing with Python to graduate students of Physics. Further information about topics: \MYhref[cyan]{http://www.slideshare.net/MojtabaKhodadadi/python-courseen-44968289}{\nolinkurl{link}}.
%I was invited by \textit{"Department of Physics"} of Isfahan University of Technology to teach a 16-hour "Scientific Computing with Python: An Introduction to Data Analysis \& High Performance Computing" to graduate students of Physics.
%}
\comment{
\subsection{Topics:}
\begin{cvcolumns}
  \cvcolumn{}{\begin{itemize} \item Introduction to Programming \item Introducing IDE: Ninja IDE, Spyder, PyCharm \item Interactive computing with IPython \item Basics of Python Programming \item Python Object Oriented Programming \item Input and Output (I/O) \end{itemize}}
  \cvcolumn{}{\begin{itemize} \item Packages and applications \item Manipulating numerical data with NumPy \item Plotting with Matplotlib in details \item Data Analysis with Pandas \item Network Analysis with NetworkX \item Introduction to HPC \end{itemize}}
\end{cvcolumns}
}

\cventry{Winter 2014}{Python Short Course for Physics Students: An introduction to Scientific Computing (14 hours) \textnormal{, \textbf{\color{red} at age 22}, for graduate students of Physics}}{Electronic and Open Learning Center, Isfahan University of Technology}{Isfahan}{Iran. \hfill  \MYhref[blue]{https://github.com/mkhm/python-scientific-computing-course-1}{\nolinkurl{github}}}{}
%I was invited by IUT \textit{Physics department} to teach Scientific Computing with Python to graduate students of Physics.
%I was invited by \textit{"Department of Physics"} of Isfahan University of Technology to teach a 14-hour "Python Short Course for Physics Students: An introduction to Scientific Computing" for graduate students of Physics.
%}
\comment{
\subsection{Topics:}
\begin{cvcolumns}
  \cvcolumn{}{\begin{itemize} \item Introduction to Programming \item Introducing IDE: Wing IDE, Pyscripter, Ninja IDE \item Basics of Python Programming \item Python Object Oriented Programming \end{itemize}}
  \cvcolumn{}{\begin{itemize} \item Input and Output (I/O) \item Packages and applications \item Manipulating numerical data with NumPy \item Plotting with Matplotlib \end{itemize}}
\end{cvcolumns}
}

\cventry{Summer 2010}{Java course: from Novice to Professional (50 hours)}{Electronic and Open Learning Center, Isfahan University of Technology}{Isfahan}{Iran}{ % \hfill \MYhref[cyan]{http://mojtabakhodadadi.physics.iut.ac.ir/content/isfahan-university-technology-iut-teacher-0}{\nolinkurl{link}}
\textbf{\color{red} At age 18}, I was invited by \textit{"Electronic and Open Learning Center"}\comment{ and \textit{"IT Center"}} to teach Java to senior and graduate students of Computer Science \& Engineering. 
\MYhref[blue]{http://mojtabakhodadadi.physics.iut.ac.ir/content/smart-bandwidth-manager-sbm}{\moreinfo}
%Further informatiom: \MYhref[cyan]{http://mojtabakhodadadi.physics.iut.ac.ir/content/isfahan-university-technology-iut-teacher-0}{\nolinkurl{link}}.
%When I was 18 years old, I was invited by \textit{"Electronic and Open Learning Center"} and \textit{"IT Center"} of Isfahan University of Technology to teach 50-hour "Java course: from Novice to Professional" for Senior and Graduate students of Computer Science \& Engineering.
}
\comment{
\subsection{Topics:} 
\begin{cvcolumns}
  \cvcolumn{}{\begin{itemize} \item Intro to Programming \& Introducing Netbeans IDE \item Basics of Java Programming \item Object Oriented Programming with Java \item Input and Output (I/O), Exception Handling \item Collections Framework \end{itemize}}
  \cvcolumn{}{\begin{itemize} \item AWT, Swing \item Generics \item Applets \item Database Connectivity (MySQL), JDBC \item Multithreading Programming \& Synchronization \end{itemize}}
\end{cvcolumns}
}


\section{Teacher Assistantship}
%\section{Teaching Experience}
%\subsection{Teacher Assistant}
\cventry{Fall 2015}{Advanced Statistical Mechanics I (graduate course)}{Instructor: \MYhref[darkblue]{http://hashemifar.iut.ac.ir/}{Dr. J. Hashemifar}}{Physics Dept., IUT}{}{}
\cventry{Fall 2015}{Advanced Statistical Mechanics II (graduate course)}{Instructor: \MYhref[darkblue]{http://shahbazi.iut.ac.ir/}{Dr. F. Shahbazi}}{Physics Dept., IUT}{}{}
\cventry{Fall 2014}{Modern Physics}{Instructor: \MYhref[darkblue]{http://fazileh.iut.ac.ir/}{Dr. F. Fazileh}}{Physics Dept., IUT}{}{}
\comment{
\comment{Course page: \MYhref[darkblue]{http://fazileh.iut.ac.ir/index_files/mp.htm}{\nolinkurl{http://fazileh.iut.ac.ir/index\_files/mp.htm}} \\ }
Responsible for helping with assignment design, grading assignments, holding recitation, and answering students questions. \comment{\hfill \MYhref[blue]{http://fazileh.iut.ac.ir/index_files/mp.htm}{\nolinkurl{link}}}
}
%\cventry{Fall 2014}{Advanced Statistical Mechanics 2}{by \MYhref[darkblue]{http://shahbazi.iut.ac.ir/}{Dr. Farhad Shahbazi}}{Physics Dept., Isfahan University of Technology}{Isfahan, Iran}{
%\comment{Course page: \MYhref[darkblue]{http://fazileh.iut.ac.ir/index_files/mp.htm}{\nolinkurl{http://fazileh.iut.ac.ir/index\_files/mp.htm}} \\ }
%Responsible for helping with assignment design, grading assignments, holding recitation, and answering students questions. \comment{\hfill \MYhref[cyan]{http://fazileh.iut.ac.ir/index_files/mp.htm}{\nolinkurl{link}}}
%}

%\subsection{Lab Instructor}
%\cventry{Spring 2014}{Physics II Lab: Electricity, Magnets \& Circuits}{}{Physics Dept., Isfahan University of Technology}{Isfahan, Iran}{}
\comment{
\subsection{Private Teaching}
\cventry{2012 -- Present}{General Physics I, II \& III}{Undergraduate students}{Isfahan}{Iran}{}
\cventry{2010 -- Present}{Java}{Senior and Graduate students of Computer Science \& Engineering}{Multithreaded Programming, Object Oriented Programming, Socket Programming, Network Programming}{Isfahan, Iran}{}
\cventry{2007 -- 2008}{Calculus, Algebra \& Physics}{}{Saadi high school}{Isfahan, Iran}{}
}


%%%%%%%%%%%%%%%%%%%%%%%%%%%%%%%%%%%
%\comment{
\section{Work Experience}
%\subsection{Vocational}
\cventry{2007 -- 2011}{Java Developer}{IT Center, Isfahan University of Technology}{Isfahan}{Iran}{%\newline{}%
%Detailed achievements:
\begin{itemize}
\item Designing, Implementing, Developing and Supporting an Electronic Library (eLibrary).
\item Testing Smart Bandwidth Management (SBM) on the IUT internet network.
\end{itemize}
}
\comment{
\subsection{Miscellaneous}
\cventry{2007 -- 2015}{Java Developer Freelancer}{}{Isfahan}{Iran}{} % \newline{}Description line 2
\comment{I have done more than \textbf{\textit{80 projects}} in Java which contain more than \textit{\textbf{700,000} lines of code.}}
%\section{Languages}
%\cvitemwithcomment{Language 1}{Skill level}{Comment}
%\cvitemwithcomment{Language 2}{Skill level}{Comment}
%\cvitemwithcomment{Language 3}{Skill level}{Comment}
%}
}
%%%%%%%%%%%%%%%%%%%%%%%%%%%%%%%%%%%%%%%%%%%%%%%%%



\section{Computer Skills}
\cvitem{Java}{8 years' experience, 
\begin{itemize}
\item \comment{\textbf{Proficient} in }Multithreaded Programming, Object Oriented Programming, Socket Programming, Network Programming, Web Scraping.
\item \comment{\textbf{Proficient} with }JSP, JSF, Hibernate, Swing, \comment{Spring framework}, JUnit framework, Log4j, \comment{GWT}, NIO, Java Sockets, JCF, RegEx, iText, JavaDJVU, PDFBox, J4L OCR, Servlets, Struts, JDBC.
\end{itemize}}
%\cvitem{Java}{\textbf{Expert} in Multithreaded Programming, Object Oriented Programming, Socket Programming, Network Programming \newline{}
%\textbf{Proficient} with JSP, JSF, Hibernate, Swing, Spring framework, JUnit framework, Log4j, GWT, NIO, iText, javadjvu, PDFBOX, J4L OCR, Servlets, Struts, JDBC.
%}
\vspace{-3ex}
\cvitem{Python}{
4 years' experience, 
\begin{itemize}
\item \comment{\textbf{Proficient} in }Scientific Computing.
\item \comment{\textbf{Proficient} with }NumPy, Pandas, Matplotlib, Scikit-learn, SciPy, VPython, NetworkX, \comment{Graph-tool, Snap, }BLZ, Blosc, Bcolz.
\end{itemize}}
\vspace{-1ex}
%\cvitem{Fortran}{
%1 year's experience}
%\vspace{-1ex}
\cvitem{Database}{MYSQL, SQLite, \comment{HSQLDB, SciDB, OpenTSDB.}}
\cvitem{Software}{Gephi, Mathematica, \comment{Matlab, }Maple.}
\cvitem{IDE \& Editor}{Netbeans, IPython, Spyder, IEP IDE, PyScripter, Ninja IDE, \comment{Spyder, PyCharm, Wing IDE, }Eclipse.}
\cvitem{Other}{Familiar with GIT, Fortran, PHP, Javascript, AJAX, C++, LaTex, HTML, XML, CSS, SQL, Wordpress, Drupal.}



\section{\textbf{Long Projects}}
%\subsection{National}
\cventry{2008 -- 2010}{Smart Bandwidth Manager (SBM)}{IT Center, Isfahan University of Technology}{Isfahan}{Iran}
{
SBM is a dynamic network bandwidth management software that I and my friend have implemented at IT Center of IUT when we were high school students. \textbf{SBM saves 48,000\$ per year for Isfahan University of Technology}. \\
\textbf{\color{projectaward}Awards:} 2\textsuperscript{nd}, 3\textsuperscript{rd} \& 4\textsuperscript{th} award in {\color{refsection}Honors \& Awards} section and the last grant in {\color{refsection}Fellowships \& Grants} section are related to this project. I started working on this project \textbf{\color{red}at age 16}.
\MYhref[blue]{http://mojtabakhodadadi.physics.iut.ac.ir/content/smart-bandwidth-manager-sbm}{\moreinfo}
}

%\subsection{Other}
\cventry{2011 -- 2013}{Electronic Content Management System (eCMS)}{Isfahan Science \& Technology Town (ISTT)}{Isfahan}{Iran}{
eCMS is a research project which its main task is extracting metadata from ebook files and checking them with LoC website for validity. It also uses Tesseract OCR technology for scanned ebooks. \\ \textbf{\color{projectaward}Awards:} 6\textsuperscript{th} grant in {\color{refsection}Fellowships \& Grants} section is related to this project.
\MYhref[blue]{http://mojtabakhodadadi.physics.iut.ac.ir/content/electronic-content-management-system-ecms}{\moreinfo}
}

\cventry{2007 -- 2010}{Electronic Library (eLibrary)}{IT Center, Isfahan University of Technology}{Isfahan}{Iran}{
eLibrary enables users to search in metadata and subject of ebooks. \textbf{eLibrary was sold to two universities each paid 8000\$ (equals to 5 years' salary of one employee in Iran).}
\MYhref[blue]{http://mojtabakhodadadi.physics.iut.ac.ir/content/electronic-library-elibrary}{\moreinfo}
}




\comment{




\section{\textbf{National Project}}
\comment{ \hfill \MYhref[blue]{http://mojtabakhodadadi.physics.iut.ac.ir/content/smart-bandwidth-manager-sbm}{\nolinkurl{link}}}
\cventry{2008 -- 2010}{Smart Bandwidth Manager (SBM)}{IT Center, Isfahan University of Technology}{Isfahan}{Iran}
{
SBM is a dynamic network bandwidth management software that I and my friend have implemented at IT Center of IUT when we were high school students. \textbf{SBM saves 48,000\$ per year for Isfahan University of Technology}.
\comment{It smartly controls and manages bandwidth of a network of sub-centers dynamically based on predefined priorities and limits of sub-centers and tries to keep total usage bandwidth on specific value.}
 \\
\textbf{\color{projectaward}Awards:} 2\textsuperscript{nd}, 3\textsuperscript{rd} \& 4\textsuperscript{th} award in {\color{refsection}Honors \& Awards} section and the last grant in {\color{refsection}Fellowships \& Grants} section are related to this project.
\MYhref[blue]{http://mojtabakhodadadi.physics.iut.ac.ir/content/smart-bandwidth-manager-sbm}{\moreinfo}
\comment{
SBM is a dynamic network bandwidth management software that smartly controls and manages bandwidth of a network of sub-centers based on predefined priorities and limits of sub-centers and tries to keep total usage bandwidth on specific value. SBM works with Cisco Internetwork Operating System (Cisco IOS) of Routers and Switches. It was tested on internet bandwidth of Isfahan University of Technology and finally \textbf{SBM saves 48,000\$ per year}. \\
Motivation: When I was high school student with my friend, we were doing research about Network Bandwidth Management at IT Center of Isfahan University of Technology. The internet usage bandwidth of the university was fixed at its maximum value (60 Mbit/sec) in office hours but in other hours decreased to 1/3 of its maximum value. We designed an efficient dynamic algorithm to assign free (unused) internet bandwidth to sub-centers which need. Afterward we implemented this algorithm to a software named \textit{''Smart Bandwidth Manager (SBM)''}.
}
}

\comment{
\cventry{2008 -- 2010}{Smart Bandwidth Manager (SBM)}{IT Center, Isfahan University of Technology}{Isfahan}{Iran}{
When I was studying mathematics \& physics at high school and pre-university school I with my friend \textit{Morteza Parsa} were doing research about Network Bandwidth Management at \textit{"IT Center"} of Isfahan University of Technology. The internet usage bandwidth of the university was fixed at its maximum value (60 Mbit/sec) in office hours but in other hours decreased to 1/3 of its maximum value. We designed an efficient dynamic algorithm to assign free (unused) internet bandwidth to nodes or sub-centers which need. Afterward we implemented this algorithm to a software named \textit{''Smart Bandwidth Manager (SBM)''}. \\
- SBM is a dynamic network bandwidth management software that smartly controls and manages bandwidth of a network of sub-centers. SBM works with Cisco Internetwork Operating System (Cisco IOS) of Routers and Switches. SBM was tested on internet bandwidth of Isfahan University of Technology. \textbf{SBM saves 120,000\$ per year}. \\
- I was also honored as \textbf{Outstanding Researcher Activity} in 2010 by Dr. Hamedani Golshan, former President of Isfahan University of Technology.
\comment{
SBM Ranked 2nd in 11th \textbf{Iran’s National Youth Kharazmi Award}, Minister of Science, Research and Technology \& Minister of Education. It was Awarded by 3 other organizations. \\
I was also honored as \textbf{Outstanding Researcher Activity} in 2010 by Dr. Hamedani Golshan, former President of Isfahan University of Technology.
}
Advantage of using SBM:
\begin{itemize}
\item Introduction to Programming
\item Introducing IDE: Wing IDE, Pyscripter, Ninja IDE
\end{itemize}
}}
\comment{
\subsection{Awards \& Grants:}
\cvlistitem{\textbf{Ranked 2nd in 11th Iran's National Youth Kharazmi Award}, Minister of Science, Research and Technology \& Minister of Education, 2009.}
\cvlistitem{\textbf{Iran's National Elites Foundation Research Grant}, Vice President for Science and Technology, Government of Iran, 2010.}
\cvlistitem{\textbf{Institute of Steel Employee Support \& Retirement Fund Award}, Presidents' Organization, 2010.}
\cvlistitem{I was honored as \textbf{Outstanding Researcher Activity}, President of Isfahan University of Technology, 2010.}
\cvlistitem{Isfahan Province Education Organization Award, Presidents' Organization, 2010.}
\cvlistitem{Research Council of Isfahan Province Education Organization Award, Presidents' Organization, 2010.}
}

\section{\textbf{Other Projects}}
\cventry{2011 -- 2013}{Electronic Content Management System (eCMS)}{Isfahan Science \& Technology Town (ISTT)}{Isfahan}{Iran}{
eCMS is a research project which its main task is extracting metadata from ebook files and checking them with LoC website for validity.\comment{ then stores these information in database in order to use by eLibrary project.} It also \comment{extracts covers from ebooks and }uses Tesseract OCR technology for scanned ebooks. \\ \textbf{\color{projectaward}Awards:} 6\textsuperscript{th} grant in {\color{refsection}Fellowships \& Grants} section is related to this project.
\MYhref[blue]{http://mojtabakhodadadi.physics.iut.ac.ir/content/electronic-content-management-system-ecms}{\moreinfo}
}
\comment{
\cventry{2011 -- 2013}{Electronic Content Management System (eCMS)}{Isfahan Science \& Technology Town (ISTT)}{Isfahan}{Iran}{
eCMS is a research project which its main task is extraction metadata such as title, authors, ISBN, publisher, year, ... from ebook files (pdf, djvu, djv, epub, mobi, ...). eCMS can extract metadata from ebook files and check them with LoC website for validity then stores these information in database in order to use by eLibrary. It also extracts covers from ebooks and use Tesseract OCR technology for scanned ebooks.
}
}
\comment{
\subsection{Grant \& Certification:}
\cvlistitem{\textbf{Isfahan Science \& Technology Town (ISTT) Research Grant}, Isfahan, Iran, 2011.}
\cvlistitem{\textbf{Isfahan Science \& Technology Town (ISTT) Accomplishment Certification}, Isfahan, Iran, 2013.}
}
\comment{
\subsection{Specifications:}
\cvlistitem{Programming Languages: Java.}
\cvlistitem{Programming Design: Object Oriented Programming, Socket Programming, Multithreaded Programming.}
\cvlistitem{Technologies \& Frameworks: PDFBox, iText, PDFRenderer, ICEpdf, Ghost4J, JavaDJVU, Swing, Java Sockets, JCF, NIO, J4L OCR, RegEx, Log4j, JDBC.}
\cvlistitem{Database: MySQL.}
}


\cventry{2007 -- 2010}{Electronic Library (eLibrary)}{IT Center, Isfahan University of Technology}{Isfahan}{Iran}{
eLibrary enables users to search in metadata and subject of ebooks. \textbf{eLibrary was sold to two universities each paid 8000\$.}
\MYhref[blue]{http://mojtabakhodadadi.physics.iut.ac.ir/content/electronic-library-elibrary}{\moreinfo}
\comment{
eLibrary enables users to search the profiles (metadata such as title, authors, ISBN, publisher, year, ...) of several thousands of ebooks and eventually view the specific ebook that they are looking for. Technically eLibrary consists of both desktop and web applications. In desktop part ''eBook Scanner'' is responsible for scanning specific hard drive to find ebooks, reading their metadatas and storing them in database. In web part "eBook Search" and "eBook Treeview" enables users to search in metadata and subject (treeview display) respectively. Some other applications was designed for managing this system in both desktop and web part. eLibrary was designed for faculty members and students of university to focus on research and get ebooks in one click. \textbf{eLibrary was sold to two universities (Tafresh University} and \textbf{Islamic Azad University of Majlesi) each paid 8000\$.}
}
}
\comment{
\subsection{Specifications:}
\cvlistitem{Programming Languages: Java, Javascript.}
\cvlistitem{Programming Design: Object Oriented Programming.}
\cvlistitem{Technologies \& Frameworks: Servlet, JSP, Swing, JavaMail, Ajax, JCF, NIO, RegEx, JDBC.}
\cvlistitem{Database: MySQL.}
}




}






\comment{
\section{Publication}
\subsection{Non-Refereed}
\cventry{Spring 2011}{''Einstein's Interpretation of Quantum Mechanics''}{Farsi translation of an article of the same name by L. E. Ballentine}{Feedback Mag., Physics Dept., Isfahan University of Technology}{Isfahan, Iran (in Persian)}{}
\cventry{Fall 2011}{''EPR Paradox''}{}{Feedback Mag., Physics Dept., Isfahan University of Technology}{Isfahan, Iran (in Persian)}{}
\cventry{Spring 2014}{''Idea of Quantum Game Theory''}{}{Feedback Mag., Physics Dept., Isfahan University of Technology}{Isfahan, Iran (in Persian)}{}
}


\comment{
\section{Course Projects \& Presentations}
\subsection{Graduate Courses at Isfahan University of Technology}
\cventry{Spring 2014}{''Facebook Networks Analysis''}{Advanced Statistical Mechanics II}{}{}{Software: Gephi, NetVizz (facebook app) \hfill \MYhref[blue]{http://www.slideshare.net/MojtabaKhodadadi/facebook-networks-analysis}{\nolinkurl{link}}} 
\cventry{Spring 2014}{''Implementing a Class for Monte Carlo Simulation by Metropolis Method in Python''}{Advanced Statistical Mechanics I}{(Python)}{}{}
%\begin{itemize}
%\item Simulating 3D Random Walk of Problem 2.2 from \textit{''Statistical Physics of Particles by Mehran Kardar, Cambridge University Press''}.
%\item Programming language: Python.
%\end{itemize}}
\cventry{Fall 2013}{''A Review of Magnetic Monopoles''}{Electrodynamics}{}{}{}
\cventry{Fall 2013}{''Molecular Dynamics Simulation of Dilute Gas \& Melting Transition and Visualization''}{Computational Physics}{(Java)}{}{}
%\begin{itemize}
%\item Programming language: Java.
%\end{itemize}}
\subsection{Undergraduate Courses at Isfahan University of Technology}
\cventry{Spring 2013}{''EPR Paradox \& Bell Inequality''}{Modern Optics}{}{}{}
\cventry{Summer 2013}{B.Sc. thesis ''A Review on Game Theory''}{under supervision of \MYhref[darkblue]{http://aghababaeisamani.iut.ac.ir/}{Dr. K. A. Samani}}{}{}{}
\cventry{Fall 2012}{''Monte Carlo Simulation of the 2D Ising Model and Visualization''}{Computer Applications in Physics}{(Python)}{}{}
%\begin{itemize}
%\item Programming language: Python.
%\end{itemize}}
\cventry{Fall 2012}{''Quantum Monte Carlo Simulation of Helium Atom''}{Computer Applications in Physics}{(Java)}{}{}
%\begin{itemize}
%\item Programming language: Java.
%\end{itemize}}
\cventry{Spring 2012}{''BCS Theory''}{Superconductivity}{}{}{}
\cventry{Spring 2012}{''Dynamics of Molecular Motors''}{Biophysics}{}{}{}
\cventry{Fall 2011}{''Terrell Rotation''}{Special Relativity}{}{}{}
\cventry{Spring 2011}{''Monte Carlo Simulation of the 2D Ising Model''}{Statistical Mechanics}{(Java)}{}{}
%\begin{itemize}
%\item Programming language: Java.
%\end{itemize}}
\cventry{Spring 2011}{''Bosons \& Fermions''}{Quantum Mechanics I}{}{}{}
\cventry{Spring 2010}{''Optical Fibers''}{Basic Physics III}{}{}{}
}





\section{Professional Membership}
\cventry{2013 -- Present}{Complex Systems Group}{}{Physics Dept., Isfahan University of Technology}{Isfahan, Iran}{}
\cventry{2009 -- Present}{Iran's National Elites Foundation (BMN)}{}{}{Government of Iran}{}
\cventry{2011 -- 2013}{Isfahan Science \& Technology Town (ISTT)}{Association of Creativity Efflorescence}{Isfahan}{Iran}{}


%\comment{
\section{Non-Academic Courses}
\cventry{Summer 2012}{Introduction to Market Research and Marketing}{}{Isfahan Science \& Technology Town (ISTT)}{Isfahan}{} %, Iran
\cventry{Fall 2011}{Ideas and Opportunities}{}{Isfahan Science \& Technology Town (ISTT)}{Isfahan, Iran}{}
\cventry{Summer 2011}{Entrepreneurship}{}{SITCO International, NIIT Isfahan Center, Iran's National Elites Foundation (BMN)}{}{}% Isfahan, Iran
\cventry{April 2010}{Security \& Hack}{}{Ashiyaneh Security Corporation}{Tehran, Iran}{}
%}


\section{Conference Organization}
\cventry{January 2015}{12th Iran National Conference on Condensed Matter Physics}{Member of Organization Committee}{Isfahan University of Technology}{Isfahan, Iran}{}
\cventry{September 2014}{International Iran Conferences on Quantum Information 2014 (IICQI)}{Member of Local Coordinating Committee}{Sharif University of Technology and Isfahan University of Technology}{Isfahan, Iran, \MYhref[blue]{http://iicqi.sharif.edu/events/iicqi-14}{\moreinfo}}{}
%{Web site: \MYhref[darkblue]{http://iicqi.sharif.edu/events/iicqi-14}{\nolinkurl{http://iicqi.sharif.edu/events/iicqi-14}}}


\section{Participation in Conferences}
\cventry{Summer 2014\comment{26/08/2014--29/08/2014}}{4th Iran's Workshop on Computational Neuroscience}{}{Physics Dept., Institute for Advanced Studies in Basic Sciences (IASBS)}{Zanjan, Iran}{}
\cventry{Winter 2013\comment{16/02/2013--20/02/2013}}{18th Special School of Physics}{}{Physics Dept., Institute for Advanced Studies in Basic Sciences (IASBS)}{Zanjan, Iran}{}
\cventry{Fall 2013\comment{25/10/2013}}{2nd IPM Conference on Soft Matter, Biological and Statistical Physics}{}{School of Physics, Institute for Research in Fundamental Sciences (IPM)}{Tehran, Iran}{}
\cventry{Fall 2013\comment{24/10/2013}}{Workshop on Physics of Polymers and Biopolymers}{}{School of Physics, Institute for Research in Fundamental Sciences (IPM)}{Tehran, Iran}{}
%\cventry{Fall 2013}{3rd Conference on Recent Progress in Foundations of Physics}{}{School of Physics, Institute for Research in Fundamental Sciences (IPM)}{Tehran, Iran}{}

\comment{

\section{References}
\begin{cvcolumns}  
  \cvcolumn{}{\textbf{Dr. Mohammad Reza Rahimi Tabar} \\ Professor of Physics \\ Institute of Physics \\ Carl von Ossietzky University, Oldenburg, Germany \\ Sharif University of Technology \\ mohammed.r.rahimi.tabar@uni-burg.de \\ P. O. Box 11365-9161}
  \cvcolumn{}{\textbf{Dr. Farhad Shahbazi} \\ Associate Professor \\ Department of Physics \\ Isfahan University of Technology \\ shahbazi@cc.iut.ac.ir \\ +9831-3391-3755}
\end{cvcolumns}
\begin{cvcolumns}  
  \cvcolumn{}{\textbf{Dr. Keivan Aghababaei Samani} \\ Associate Professor \\ Department of Physics \\ Isfahan University of Technology \\ samani@cc.iut.ac.ir \\ +9831-3391-3763}
  \cvcolumn{}{\textbf{Dr. Massoud Reza Hashemi} \\ Associate Professor \\ Department of Electrical \& Computer Engineering \\ Isfahan University of Technology \\ hashemim@cc.iut.ac.ir \\ +9831-3391-5395}
\end{cvcolumns}
\begin{cvcolumns}  
  \cvcolumn{}{\textbf{Dr. Mojtaba Alaei} \\ Assistant Professor \\ Department of Physics \\ Isfahan University of Technology \\ m.alaei@cc.iut.ac.ir \\ +9831-3391-3704}
  \cvcolumn{}{\textbf{Dr. Vahid Salari} \\ Assistant Professor \\ Department of Physics \\ Isfahan University of Technology \\ vahidsalari@cc.iut.ac.ir \\ +9831-3391-3709}
\end{cvcolumns}

}
\comment{


\section{Extra 1}
\cvlistitem{Item 1}
\cvlistitem{Item 2}
\cvlistitem{Item 3. This item is particularly long and therefore normally spans over several lines. Did you notice the indentation when the line wraps?}




\section{References}
\begin{cvcolumns}
  \cvcolumn{Category 1}{\begin{itemize}\item Person 1\item Person 2\item Person 3\end{itemize}}
  \cvcolumn{Category 2}{Amongst others:\begin{itemize}\item Person 1, and\item Person 2\end{itemize}(more upon request)}
  \cvcolumn[0.5]{All the rest \& some more}{\textit{That} person, and \textbf{those} also (all available upon request).}
\end{cvcolumns}



% Publications from a BibTeX file without multibib
%  for numerical labels: \renewcommand{\bibliographyitemlabel}{\@biblabel{\arabic{enumiv}}}% CONSIDER MERGING WITH PREAMBLE PART
%  to redefine the heading string ("Publications"): \renewcommand{\refname}{Articles}
\nocite{*}
\bibliographystyle{plain}
\bibliography{publications}                        % 'publications' is the name of a BibTeX file

% Publications from a BibTeX file using the multibib package
%\section{Publications}
%\nocitebook{book1,book2}
%\bibliographystylebook{plain}
%\bibliographybook{publications}                   % 'publications' is the name of a BibTeX file
%\nocitemisc{misc1,misc2,misc3}
%\bibliographystylemisc{plain}
%\bibliographymisc{publications}                   % 'publications' is the name of a BibTeX file

\clearpage
%-----       letter       ---------------------------------------------------------
% recipient data
\recipient{Company Recruitment team}{Company, Inc.\\123 somestreet\\some city}
\date{January 01, 1984}
\opening{Dear Sir or Madam,}
\closing{Yours faithfully,}
\enclosure[Attached]{curriculum vit\ae{}}          % use an optional argument to use a string other than "Enclosure", or redefine \enclname
\makelettertitle

Lorem ipsum dolor sit amet, consectetur adipiscing elit. Duis ullamcorper neque sit amet lectus facilisis sed luctus nisl iaculis. Vivamus at neque arcu, sed tempor quam. Curabitur pharetra tincidunt tincidunt. Morbi volutpat feugiat mauris, quis tempor neque vehicula volutpat. Duis tristique justo vel massa fermentum accumsan. Mauris ante elit, feugiat vestibulum tempor eget, eleifend ac ipsum. Donec scelerisque lobortis ipsum eu vestibulum. Pellentesque vel massa at felis accumsan rhoncus.

Suspendisse commodo, massa eu congue tincidunt, elit mauris pellentesque orci, cursus tempor odio nisl euismod augue. Aliquam adipiscing nibh ut odio sodales et pulvinar tortor laoreet. Mauris a accumsan ligula. Class aptent taciti sociosqu ad litora torquent per conubia nostra, per inceptos himenaeos. Suspendisse vulputate sem vehicula ipsum varius nec tempus dui dapibus. Phasellus et est urna, ut auctor erat. Sed tincidunt odio id odio aliquam mattis. Donec sapien nulla, feugiat eget adipiscing sit amet, lacinia ut dolor. Phasellus tincidunt, leo a fringilla consectetur, felis diam aliquam urna, vitae aliquet lectus orci nec velit. Vivamus dapibus varius blandit.

Duis sit amet magna ante, at sodales diam. Aenean consectetur porta risus et sagittis. Ut interdum, enim varius pellentesque tincidunt, magna libero sodales tortor, ut fermentum nunc metus a ante. Vivamus odio leo, tincidunt eu luctus ut, sollicitudin sit amet metus. Nunc sed orci lectus. Ut sodales magna sed velit volutpat sit amet pulvinar diam venenatis.

Albert Einstein discovered that $e=mc^2$ in 1905.

\[ e=\lim_{n \to \infty} \left(1+\frac{1}{n}\right)^n \]

\makeletterclosing

%\clearpage\end{CJK*}                              % if you are typesetting your resume in Chinese using CJK; the \clearpage is required for fancyhdr to work correctly with CJK, though it kills the page numbering by making \lastpage undefined




}
\end{document}
